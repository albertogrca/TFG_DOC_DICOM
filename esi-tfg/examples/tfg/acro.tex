\chapter{Listado de acrónimos}

{\small
\begin{acronym}[XXXXXXXX]
  \Acro{GNU}     {\acs{GNU} is Not Unix}
  \acro{OO}      {Orientación a Objetos}
  \acro{RPC}     {Remote Procedure Call}
  \acro{TFG}     {Trabajo de Fin de Grado}
  \acro{DICOM}   {Digital Imaging and Communications on Medicine}
  \acro{ACR}     {American College of Radiology}
  \acro{NEMA}    {National Electrical Manufactures Association}
  \acro{TIC}     {Tecnologías de la información y de la Comunicación}
  \acro{IHE}     {Integrating the Healthcare Enterprise}
  \acro{PACS}    {Picture Archiving and Comunications Systems}
  \acro{ER}      {Entidad-Relación}
  \acro{IE}      {Information Entity (Entidad de Información}
  \acro{IOD}     {Information Object Definition}
  \acro{SCU}     {Service Class User}
  \acro{SCP}     {Service Class Provider}
  \acro{DIMSE}   {DICOM Message Service Element}
  \acro{DIMCO UL}   {DICOM Upper Layer}
\end{acronym}
}


% \ac{OO}   la primera vez \acf, después \acs
% \acs{OO}  short: OO
% \acf{OO}  full : Object Oriented (OO)
% \acl{OO}  large: Object Oriented
% \acx{OO}         OO (Object Oriented)

% usa \Acro cuando no debe aparecer nunca expandido en el texto

% Local variables:
%   TeX-master: "main.tex"
% End:
