\chapter{Resumen}

A día de hoy, los avances en las imágenes clínicas han supuesto una gran mejora en el diagnóstico, tanto en la eficacia, como en al mejora de la calidad de las imágenes. Uno de los avances de la imagen clínica, es la posibilidad de almacenar información referente a las exploraciones médicas como información adjunta a las imágenes médicas obtenidas. Esta información se encuentra codificada de acuerdo a las especificaciones del estándar DICOM. Todo esto genera gran cantidad de datos que se debe tratar de forma correcta y rápida, para ello los hospitales utilizan los sistemas PACS.

Estos sistemas proporcionan el poder de almacenar y gestionar todo este gran volumen de datos, a través de la interconexión de diferentes equipos médicos, bases de datos, estaciones de visualización y dispositivos de impresión, conforme a las especificaciones de DICOM y las recomendaciones de IHE. 

Para el desarrollo de este Trabajo de Fin de Grado (a partir de ahora llamado TFG por simplicidad) se tomará como objetivo la creación de una herramienta fácil y rápida de instalar, que imite la funcionalidad de un PACS, pero a menor escala, ya que el objetivo de esta herramienta no esta orientada a los hospitales, sino a los servicios clínicos pequeños, que por la gran cantidad de recursos que cuesta implantar un sistema PACS, tanto económicos como humanos, no pueden implantarlo.

Este TFG ha sido desarrollado siguiendo estándares y tecnologías libres, que permiten su extensibilidad y reutilización. Además, la arquitectura está basada en el uso de patrones de diseño y organizada en módulos independientes con responsabilidades bien definidas, proporcionando una implementación poco acoplada y muy cohesionada.


\chapter{Abstract}

English version of the previous page.
